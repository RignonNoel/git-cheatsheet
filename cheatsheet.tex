\documentclass[10pt,landscape]{article}
\usepackage{multicol}
\usepackage{calc}
\usepackage{ifthen}
\usepackage[landscape]{geometry}
\usepackage{titlesec}
\usepackage{sectsty}

\sectionfont{\vspace{0.5em}\fontsize{12}{15}\selectfont}

% French

\usepackage[utf8]{inputenc}
\usepackage[T1]{fontenc}
\usepackage{ae,aeguill}
\usepackage[canadian]{babel}

\usepackage{listings}
\lstset{basicstyle=\ttfamily,
  showstringspaces=false,
  commentstyle=\color{red},
  keywordstyle=\color{blue}
}

% This sets page margins to .5 inch if using letter paper, and to 1cm
% if using A4 paper. (This probably isn't strictly necessary.)
% If using another size paper, use default 1cm margins.
\ifthenelse{\lengthtest { \paperwidth = 11in}}
	{ \geometry{top=.5in,left=.5in,right=.5in,bottom=.5in} }
	{\ifthenelse{ \lengthtest{ \paperwidth = 297mm}}
		{\geometry{top=1cm,left=1cm,right=1cm,bottom=1cm} }
		{\geometry{top=1cm,left=1cm,right=1cm,bottom=1cm} }
	}

% Turn off header and footer
\pagestyle{empty}


% Redefine section commands to use less space
\makeatletter
\titlespacing{\section}{0pt}{1ex plus 1ex minus 1ex}{1ex plus 1ex minus 1ex}

\makeatother

% Don't print section numbers
\setcounter{secnumdepth}{0}

\setlength{\parindent}{0pt}
\setlength{\parskip}{0pt plus 1ex}


% -----------------------------------------------------------------------

\begin{document}

\raggedright
\footnotesize
\begin{multicols}{3}


% multicol parameters
% These lengths are set only within the two main columns
%\setlength{\columnseprule}{0.25pt}
\setlength{\premulticols}{1pt}
\setlength{\postmulticols}{1pt}
\setlength{\multicolsep}{1pt}
\setlength{\columnsep}{2pt}

\begin{center}
     \Large{\textbf{Rappel Git}} \\
\end{center}

%%%%%%%%%%%%%%%% Configuration initiale %%%%%%%%%%%%%%%%
\section{Configuration initiale}
\begin{lstlisting}[language=bash]
git config --global user.name "[name]"
\end{lstlisting}
Configure le nom d’utilisateur associé aux futurs commits.
\begin{lstlisting}[language=bash]
git config --global user.email "[email]"
\end{lstlisting}
Configure le courriel associer au futur commits.


%%%%%%%%%%%%%%%% Créer un nouveau dépôt (repository) %%%%%%%%%%%%%%%%
\section{Créer un nouveau dépôt (repository)}
\begin{lstlisting}[language=bash]
git init [project-name]
\end{lstlisting}
Crée un nouveau dépôt (repository) en local.\\
\begin{lstlisting}[language=bash]
git clone [url]
\end{lstlisting}
Télécharge un dépôt (repository) et tout son historique de version.


%%%%%%%%%%%%%%%% Créer un nouveau dépôt (repository) %%%%%%%%%%%%%%%%
\section{Gestion des fichiers}
\begin{lstlisting}[language=bash]
git rm [file]
\end{lstlisting}
Supprime le fichier et notifie la suppression pour le prochain commit.\\
\begin{lstlisting}[language=bash]
git rm --cached [file]
\end{lstlisting}
Supprime le fichier de la liste des fichiers suivis par git, mais ne touche pas au fichier local.\\
\begin{lstlisting}[language=bash]
git mv [file-original] [file-renamed]
\end{lstlisting}
Renomme le fichier et notifie le changement pour le\\
prochain commit.

%%%%%%%%%%%%%%%% Supprimer/Restaurer temporairement des changements %%%%%%%%%%%%%%%%
\section{Supprimer/Restaurer temporairement des changements}
\begin{lstlisting}[language=bash]
git stash
\end{lstlisting}
Cache temporairement tous les changements non commités dans un paquet.
\begin{lstlisting}[language=bash]
git stash pop
\end{lstlisting}
Restaure le dernier paquet de changement caché.
\begin{lstlisting}[language=bash]
git stash list
\end{lstlisting}
Liste tous les paquets de changement caché.
\begin{lstlisting}[language=bash]
git stash drop
\end{lstlisting}
Supprime le dernier paquet de changement caché.

%%%%%%%%%%%%%%%% Effectuer des changements %%%%%%%%%%%%%%%%
\section{Effectuer des changements}
\begin{lstlisting}[language=bash]
git status
\end{lstlisting}
Liste tous les nouveaux fichiers et tous les fichiers ayant subi un changement qui vont être présents dans la nouvelle version.
\begin{lstlisting}[language=bash]
git add [file]
\end{lstlisting}
Ajoute les changements actuels du fichier à la liste des changements devant être dans la nouvelle version.
\begin{lstlisting}[language=bash]
git reset [file]
\end{lstlisting}
Enlève le fichier de la liste des changements devant être dans la nouvelle version.
\begin{lstlisting}[language=bash]
git commit -m"[descriptive message]"
\end{lstlisting}
Sauvegarde la nouvelle version de manière permanente dans le gestionnaire de version.

%%%%%%%%%%%%%%%% Les branches %%%%%%%%%%%%%%%%
\section{Les branches}
\begin{lstlisting}[language=bash]
git branch
\end{lstlisting}
Liste toutes les branches présentes localement sur le dépôt (repository)
\begin{lstlisting}[language=bash]
git branch [branch-name]
\end{lstlisting}
Crée une nouvelle branche
\begin{lstlisting}[language=bash]
git checkout [branch-name]
\end{lstlisting}
Déplace l'utilisateur vers la branche spécifiée et mets à jour les fichiers présents dans le dépôt.
\begin{lstlisting}[language=bash]
git merge [branch-name]
\end{lstlisting}
Combine la branche spéifiée avec la branche courante.
\begin{lstlisting}[language=bash]
git branch -d [branch-name]
\end{lstlisting}
Supprime la branche spécifiée.

%%%%%%%%%%%%%%%% Analyser l'historique de version %%%%%%%%%%%%%%%%
\section{Analyser l'historique de version}
\begin{lstlisting}[language=bash]
git log
\end{lstlisting}
Liste tout les commits de la branche courante.
\begin{lstlisting}[language=bash]
git log --follow [file]
\end{lstlisting}
Liste tous les commits relatifs au fichier, incluant les changements de nom.
\begin{lstlisting}[language=bash]
git diff [first-branch]...[second-branch]
\end{lstlisting}
Montre les différences de contenu entre deux branches.
\begin{lstlisting}[language=bash]
git show [commit]
\end{lstlisting}
Montre le détail du commit spécifié (changement, nom, heure, auteur, ..).

%%%%%%%%%%%%%%%% Modifier des commits %%%%%%%%%%%%%%%%
\section{Modifier des commits}
\begin{lstlisting}[language=bash]
git reset [commit]
\end{lstlisting}
Annule tous les commits effectués après le commit spécifié, cela préserve donc les changements.
\begin{lstlisting}[language=bash]
git reset --hard [commit]
\end{lstlisting}
Supprime tous les commits effectués après le commit spécifié, les changements sont donc supprimés

%%%%%%%%%%%%%%%% Synchroniser les changements %%%%%%%%%%%%%%%%
\section{Synchroniser les changements}
\begin{lstlisting}[language=bash]
git fetch [remote]
\end{lstlisting}
Synchronise en local l'historique du dépôt (repository) distant.
\begin{lstlisting}[language=bash]
git merge [remote]/[branch]
\end{lstlisting}
Combine la branche distante avec la branche locale.
\begin{lstlisting}[language=bash]
git push [remote] [branch]
\end{lstlisting}
Téléverse tous les commits de la branche locale sur le dépôt (repository) distant.
\begin{lstlisting}[language=bash]
git pull [remote] [branch]
\end{lstlisting}
Télécharge tout les nouveaux changements présent sur le dépôt et les applique en local.

\end{multicols}
\end{document}
